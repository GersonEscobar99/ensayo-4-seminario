\documentclass[12pt]{article}
\usepackage[utf8]{inputenc}
\usepackage[spanish]{babel}
\usepackage{csquotes}
\usepackage[backend=biber,style=apa,sorting=nyt]{biblatex}
\usepackage{times}
\usepackage{geometry}
\usepackage{titling}
\setlength{\droptitle}{-2cm}
\geometry{letterpaper, margin=2.4cm}

\addbibresource{referencias_oauth2.bib}

\setlength{\parskip}{0pt}
\setlength{\parindent}{0pt}  
\renewcommand{\baselinestretch}{1.0}

\title{\textbf{\MakeUppercase{OAUTH 2.0}}}
\author{\textit{G. M. Escobar Aguilar}\\
\textit{7690-20-3975 Universidad Mariano Gálvez}\\
\textit{Seminario de tecnologías de información}\\
\textit{gescobara3@miumg.edu.gt}}
\date{}

\begin{document}
\maketitle

\textbf{2. Resúmen}  
\\
OAuth 2.0 es un marco de autorización ampliamente adoptado que permite a las aplicaciones obtener acceso limitado a recursos protegidos sin necesidad de exponer credenciales sensibles. Este artículo describe los principios de OAuth 2.0, sus principales roles y flujos de autorización, así como las ventajas y desafíos que plantea su implementación. Se discuten casos de uso comunes en aplicaciones web y móviles, junto con las implicaciones en materia de seguridad y experiencia de usuario. En conclusión, OAuth 2.0 se presenta como un estándar clave en el ecosistema digital actual, fundamental para garantizar accesos controlados en un entorno cada vez más interconectado.

\textbf{3. Palabras claves:}  
\\
OAuth 2.0, autorización, autenticación, seguridad, API

\textbf{4. Desarrollo del tema}  
\\
OAuth 2.0, definido por la IETF en el RFC 6749, se utiliza como un protocolo de autorización que permite a los usuarios otorgar acceso limitado a aplicaciones de terceros sin compartir sus contraseñas. En lugar de entregar credenciales directas, el sistema utiliza \textbf{tokens de acceso} de corta duración, que pueden renovarse mediante \textbf{tokens de actualización} (Hardt, 2012).  

Los roles principales en OAuth 2.0 son:  
- \textbf{Resource Owner (propietario del recurso):} el usuario que concede acceso a sus datos.  
- \textbf{Client (cliente):} la aplicación que solicita acceso.  
- \textbf{Authorization Server (servidor de autorización):} valida la identidad y emite tokens.  
- \textbf{Resource Server (servidor de recursos):} contiene los datos protegidos y valida los tokens.  

Existen distintos flujos de autorización según el tipo de aplicación:  
- \textbf{Authorization Code Flow:} el más seguro, utilizado por aplicaciones web con servidor backend.  
- \textbf{Implicit Flow:} pensado para aplicaciones de una sola página, aunque actualmente se desaconseja por motivos de seguridad.  
- \textbf{Resource Owner Password Flow:} otorga acceso directo usando credenciales del usuario, pero se limita a casos muy específicos.  
- \textbf{Client Credentials Flow:} usado cuando una aplicación accede a sus propios recursos, sin un usuario humano.  

OAuth 2.0 se ha convertido en un estándar de facto en servicios como Google, Facebook o GitHub, ya que permite integrar autenticación y autorización en ecosistemas complejos de manera flexible y relativamente sencilla. Sin embargo, no está exento de riesgos, ya que una implementación incorrecta puede generar vulnerabilidades graves, como la exposición de tokens.

\textbf{5. Observaciones y comentarios}  
\\
OAuth 2.0 no es un protocolo de autenticación en sí mismo, sino de autorización. Para autenticación suele complementarse con \textbf{OpenID Connect}, que agrega una capa de identidad. En la práctica, muchas organizaciones confunden ambos conceptos, lo cual puede generar configuraciones inseguras.

\textbf{6. Conclusiones}  
\\
1. OAuth 2.0 es un marco de autorización esencial para aplicaciones modernas conectadas a internet.  
2. Sus flujos de autorización permiten adaptar la seguridad a distintos escenarios de uso.  
3. Una implementación deficiente puede comprometer la seguridad, por lo que requiere buenas prácticas y auditorías constantes.  
4. Combinado con OpenID Connect, constituye la base de autenticación y autorización en ecosistemas distribuidos.  

\textbf{7. Bibliografía}  
\\
Auth0. (2023). OAuth 2.0 and OpenID Connect Overview. https ://auth0.com/docs/get-started/authentication-and-authorization-flow
\\
Hardt, D. (2012). The OAuth 2.0 Authorization Framework (inf. t´ec. N.o RFC 6749). Internet Engi-neering Task Force (IETF). https://www.rfc-editor.org/rfc/rfc6749
\\
Inc., O. (2023). What is OAuth 2.0? https://www.okta.com/identity-101/what-is-oauth-2-0/

\vspace{0.5cm}
\noindent URL del repositorio Git: \texttt{https://github.com/GersonEscobar99/ensayo-4-seminario}

\end{document}
